\documentclass[10pt]{scrartcl}
\usepackage[utf8]{inputenc}
\usepackage{lmodern}
\usepackage{amsmath, amssymb}
\usepackage{a4wide}
\usepackage{placeins}
\usepackage{graphicx}
\usepackage{hyperref}
\hypersetup{
    colorlinks,
    linkcolor   = blue,
    urlcolor    = blue,
    citecolor   = blue,
    anchorcolor = blue
}

%opening
\title{
Tuner \\
{\small frequency analysis of the microphone input suitable to tune instruments}
}
\author{by Richard Hartmann}

\renewcommand{\i}{\mathrm{i}}
\newcommand*{\defeq}{\mathrel{\vcenter{\baselineskip0.5ex \lineskiplimit0pt
                     \hbox{\scriptsize.}\hbox{\scriptsize.}}}=}
\newcommand*{\eqdef}{=\mathrel{\vcenter{\baselineskip0.5ex \lineskiplimit0pt
                     \hbox{\scriptsize.}\hbox{\scriptsize.}}}}
                     
\newcommand{\kll}[1]{\textit{#1}{,}{,}}
\newcommand{\kl}[1]{\textit{#1}{,}}
\renewcommand{\k}[1]{\textit{#1}}
\newcommand{\kh}[1]{\textit{#1}{'}}
\newcommand{\khh}[1]{\textit{#1}{'}{'}}
\newcommand{\khhh}[1]{\textit{#1}{'}{'}{'}}
\newcommand{\khhhh}[1]{\textit{#1}{'}{'}{'}{'}}
\newcommand{\khhhhh}[1]{\textit{#1}{'}{'}{'}{'}{'}}

\begin{document}

\maketitle
\newpage

\tableofcontents
\newpage

\section{Theory}

\subsection{Piano Keys -- Well-Tempered Tuning}

Starting at some reference key, e.g. the key \k{a} with a frequency of 440Hz, the key an octave higher has twice that frequency. 
Thus, the key an octave lower, half the frequency.
In twelve-tone equal temperament an octave is made up of twelve steps with frequencies that are equally spaced in a logarithmic scale.
So the frequency of the key, labeled in steps above/below the reference key \k{a} at frequency  440Hz is given by
\begin{equation}
    f_i = 2^{i/12} 440 \mathrm{Hz} \; .
\end{equation}
On a standard piano with 88 keys the reference key \k{a} is the 49th key.
So labeling the key with index $n$ from low to high yields the alternative expression
\begin{equation}
    f_n = 2^{(n-49)/12} 440 \mathrm{Hz} \; .
\end{equation}

This means, that the lowest key, i.e. the \kll{a} has a frequency of $f_1 = 27.5$Hz which amount to a period of $T_1 = 1/f_1 = 36.4\mathrm{ms}$. 
For the highest key \khhhh{c} it follows $f_{88} = 4186$Hz and $T = 0.239\mathrm{ms}$.






\end{document}
